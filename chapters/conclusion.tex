\section{Conclusion}
Based on the experiments performed in the research, it is evident that transfer-learning using the domain compatible histopathological dataset provides high accuracy classification of the  cytological image sample. However, applying the partial fine-tuning to the context of the transfer learning is meaningful as compared to using the complete fine-tuning. The study also concludes that employing distinct CNN model architecture does not produce significant differences in the results except for the VGG-19 which performed worse than other models used in the context of transfer learning. The proposed framework yields 6-17\% improved accuracy when compared with the existing traditional machine learning solutions and  7\% compared to the solutions based on CNN methods.  Furthermore, the result of the study eliminates the needs for huge amounts of annotated data for the binary classification of the cytological images while providing the optimal results. The investigation has used a 17 times smaller dataset compared to the existing system while providing 3 per cent optimal results in terms of accuracy.

\section{Future Works}
In order to improve the existing framework, the speaker is looking forward to applying the multi-source domain adaptation techniques which will result in the minimizing the domain-shift between the source and target domain. Furthermore, the utilization of the light-weight network is also taken into consideration in order to  diminish the system’s dependence.  Atlast, the investigation will further be extended to be performed on the image-level scope instead of patch-level to evaluate the systems performance in classification of cancer malignancies in the patients. 